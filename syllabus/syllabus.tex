\documentclass[12pt]{article}

\usepackage[margin=0.93in]{geometry}
\input{../../syllabi/preamble}

%%edit these!!
\newcommand{\coursedept}{Math}
\newcommand{\coursenumber}{340}
\newcommand{\professorname}{Adam Kapelner, PhD.}
\newcommand{\professorcontactinfo}{kapelner@qc.cuny.edu}
\newcommand{\professoroffice}{604 Kiely Hall}
\newcommand{\sixhundredsection}{640}
\newcommand{\coursenumbercrosslisted}{/ \sixhundredsection}
\newcommand{\semester}{Fall}
\newcommand{\numcredits}{4}
\newcommand{\lectimeandloc}{e.g. Mon and Wed 4:30 -- 6:20PM / KY258}
\newcommand{\requiredlabtimeandloc}{}
\newcommand{\tataofficehourtimeandloc}{} %TA / TA Office Hours / Loc & Kennly Weerasinghe / see \coursewebpagelink
\newcommand{\numtheoryhws}{6--9}
\newcommand{\lastdatetimetohandinhomeworks}{Dec 15 at noon}
\newcommand{\midtermonedateandlocation}{Thursday, October 7 on zoom during class time}
\newcommand{\midtermtwodateandlocation}{Thursday, November 11 on zoom during class time}
\newcommand{\finaldateandlocation}{TBD but on zoom}

%%%unchanged
\newcommand{\coursewebpageurl}{https://github.com/kapelner/QC_\coursedept_\coursenumber_\semester_\the\year}
\newcommand{\coursewebpagelink}{\href{\coursewebpageurl}{course homepage}}
\newcommand{\slackurl}{https://QC\coursedept\coursenumber\semester\the\year.slack.com/}
\newcommand{\slacklink}{\href{\slackurl}{slack}}

\input{../../syllabi/_header}



\section*{Course Overview}

MATH 340 / 640. Probability Theory for Data Science and Statistics. 4 hr.; 4 cr. Prereq.: 241. Coreq.: MATH 201 and 231. Convolutions, multivariate transformation of variables, the poisson process, beta, logistic, laplace, Weibull distributions, characteristic functions, central limit theorem, Cochran’s Theorem, the multivariate normal, chi-squared, T, F, Cauchy distributions, vector random variables, covariance matrices, multinomial, multivariate normal, order statistics, law of large numbers, famous inequalities, convergence in distribution and probability, laws of large numbers, Slutsky’s Theorem, optimization including stochastic gradient descent, Markov chains, Gibbs sampling. Probability computation using modern software. Special topics. Not open to students who are taking or who have received credit for MATH 340. Students cannot receive credit for more both: MATH 340, and 640. Fall, Spring. \pagebreak

\input{../../syllabi/_DSS_core}

Examining the above, we note that MATH 340 can be taken as a standalone course as a higher-level math elective. It can be thought of as advanced probability following MATH 241. This course is also recommended for students who are considering an actuarial career as this course provides great practice for Exam P. \pagebreak

\section*{Tentative Day-by-Day Schedule}

Lectures and their topics with rough time estimates per topic are below:

\begin{enumerate}
\item[D1 - Lec 1] [10min] Review of discrete random variables (rvs), support; [10min] the Bernoulli rv, parameter space, degenerate rv, indicator functions; [10min] old-style probability mass functions (PMFs), new-style PMFs using the indicator function; [10min] cumulative distribution functions (CDFs), survival functions, [10min] vector rvs, joint mass functions (JMFs), independence, identical distributedness, iid; [25min] sum of two Bernoulli rvs, convolution operator, tree diagram for two rvs, conditional probability, marginal probability; [10min] Plotting the PMF of the convolution; [20min] discrete convolution formulas (general, independent rvs, iid rvs, old and new style), convolution support

\item[D2 - Lec 2] [5min] definition of data as realization from rv's [5min] Uniform discrete rv  [5min] defining combinatorial terms with indicator functions; [5min] visualizing convolutions as passing PMFs \qu{through each other} [35min] Binomial rv with PMF derivation, Pascal's identity [40min] Sequences of rv's, derivation of the geometric rv as a \qu{waiting time} or \qu{survival} distribution, geometric rv PMF, geometric series [30min] convolution of geometric rvs, the negative binomial rv

% [25min] derivation of the Poisson rv, convolution of Poisson rvs; 
\item[D3 - Lec 3] [30min] Strategy to compute $\mathbb{P}(X > Y)$ for two rvs $X, Y$, the sum reindexing trick [15min] CDF / survival function of geometric rv [30min] derivation of the exponential rv as a limit of geometric rv's, function limits [15min] definition of continuous rvs, PDF's, PDF rules, JDF's [10min] Uniform rv [30min] derivation of convolution formulas for continuous rvs;

\item[D4 - Lec 4] [35min] Convolution of two iid standard uniform rv's using both derivation methods [30min] derivation of the Erlang rv as a sum of iid exponential rv's [20min] review of complex numbers, Euler's identity  [15min] definition of forward and inverse Fourier integral transform [20min] definition and properties of characteristic functions  
 
\item[D5 - Lec 5] [10min] demo of Fourier analysis with singing head and chest voices and the same note on the electric piano; [40min] definition of characteristic functions (chfs) and properties 0, 1, ..., 8 [20min] chf's for Bernoulli, Binomial, Geometric, Exponential [25min] Proof of the very weak Law of Large Numbers (LLN) [20min] Setup of the central limit theorem (CLT), standardization of rv's

%[10min] derivation of the chf for the Gamma rv and proof that two Gamma rvs convolved is a Gamma rv; 

%[25min] multinomial rv, multichoose notation; [20min] proof that multinomial's marginals are Binomial 

%\item[D4] [30min] Proof that conditional multinomials are multinomial, indicator functions for undefinedness; [25min] review of expectation, variance, standard deviation, covariance of two rvs, covariance rules; [5min] expectation of vector rvs; [15min] variance-covariance matrix (varcov); [5min] rules of expectation of vector rvs; [15min] rules of varcov and quadratic forms; [5min] Markowitz optimal portfolio theory application

\item[D6 - Lec 6] [45min] Proof of the central limit theorem (CLT): derivation of the limiting characteristic function, inversion of the limitiing ch.f [30min] properties of the standard normal, the general normal rv, the normal CDF ($\Phi$ function), approximate distributions due to asymptotic convergence [30min] applications of the CLT [5min] intro to Multinomial rv

\item[D7 - Lec 7] [45min] derivation of JMF of multinomial rv, support, parameter space, humpty-dumpty rule [20min] proof that marginal distribution of multinomial is binomial [30min] derivation of covariance, covariance properties, [25min] proof of Cauchy-Schwartz inequality and the covariance inequality

%\item[D7] [30min] gamma function, incomplete gamma functions, regularized incomplete gamma function, gamma function properties and identities; [20min] CDF of Erlang, CDF of Poisson; [15min] Poisson Process; [10min] Gamma rv and Extended Negative Binomial rv; [15min] transformations of discrete rvs to derive PMFs


\item[D8] \inblue{Midterm I Review (Wednesday, Sept 27)}
\item[D9] \inred{Midterm I (Monday, Oct 2)}


\item[D10 - Lec 8]  [5min] expectation of vector rv's, special case of multinomial, expectation of matrix rv's [15min] definition of variance-covariance matrix, special case of independence and iid [25min] proof of covariance of components of the multinomial rv [20min] proof of multinomial being the conditional distribution within the multinomial [20min] Markov's inequality [20min] Markov's corrolaries including Chebyshev's inequality

\rule{8cm}{0.4pt} \inblue{Done updating until here}

\item[D11 - Lec 9] [20min] derivation of the Pareto rv PDF, CDF and quantile function; [30min] applied project: use Pareto to model land ownership; [15min] derivation of the Laplace rv; [10min] error distributions; [15min] derivation of the Weibull rv, memorylessness and the Weibull modulus; [20min] applied project: use Weibull to model human lifespans

\item[D12 - Lec 10] [50min] order statistics, CDF/PDF of minimum and maximum, CDF of arbitrary order statistics; [15min] derivation of the PDF of arbitrary order statistics; [15min] order statistics of iid Uniform rvs to derive the Beta rv; [15min] kernels of PMFs/PDFs; [15min] proof that convolution of Gamma rvs is a Gamma rv

\item[D13 - Lec 11]  [15min] the beta function, incomplete beta function, regularized beta function, the Gamma-Beta function identity, the CDF of the Beta rv; [35min] Multivariate transformations of rvs, Jacobian determinants; [10min] PDF of ratio of rv formulas; [10min] PDF of proportion of sum of rv formulas; [10min] proof that the proportion of sum of two Gamma rvs is a Beta rv; [10min] proof that the ratio of two Gamma rvs is a BetaPrime rv

\item[D14 - Lec 12]  [35min] mixture and compound rvs; [10min] derivation that the Poisson rv compounded with a Gamma rv is a ExtendedNegativeBinomial rv; [15min] derivation that the Binomial rv compounded with a Beta rv is the BetaBinomial rv; [15min] review of complex numbers, proof of Euler's famous formula relating 0, 1, $e$, $i$ and $\pi$; [20min] basics of Fourier transform / analysis and Fourier inverse transform / synthesis, L1 integrable functions, statement of the Fourier inversion theorem

\item[D15 - Lec 13] [20min] demo of Fourier analysis with singing head and chest voices and the same note on the electric piano; [40min] definition of characteristic functions (chfs) and properties 0, 1, ..., 8, definition of convergence in distribution for Levy's Continuity theorem; [5min] definition of moment generating functions (mgfs) and properties 0, 1, \ldots, 4; [10min] derivation of the chf for the Gamma rv and proof that two Gamma rvs convolved is a Gamma rv; [25min] Proof of the central limit theorem (CLT): derivation of the limiting characteristic function
 

\item[D16 - Lec 14] [25min] Proof of the central limit theorem (CLT): inverting the limiting characteristic function to derive the PDF of the standard Normal rv; [15min] derivation of the general Normal rv, its moments, its chf; [10min] proof that the convolution of two independent Normal rvs is a Normal rv; [5min] derivation of the LogNormal rv and an application in portfolio theory; [25min] derivation of the $\chi^2_n$ rv and its equivalence to a Gamma rv, derivation that scaled Gamma rvs are Gamma rvs, derivation of the $\chi_n$ rv

\item[D17 - Lec 15] [20min] derivation of the F rv as the ratio of scaled $\chi^2_n$ rvs; [20min] derivation of Student's T rv as the ratio of a Normal rv to a scaled $\chi$ rv; [10min] derivation of the standard and general Cauchy rv; [5min] Proof that the expectation of the Cauchy rv doesn't exist; [10min] the physicist's derivation of the Cauchy rv; [15min] derivation that the scaled sample variance is a $\chi^2_n$ rv; [15min] quadratic forms embedded in $\chi^2_n$; [15min] Cochran's theorem introduction


\item[D18] \inblue{Midterm II Review (Wednesday, Nov 1)}
\item[D19] \inred{Midterm II (Monday, Nov 6)}


\item[D20 - Lec 16] [30min] Cochran's thm proof including simultaneous diagonalization from linear algebra; [15min] the independent of sample variance and sample average in the case of iid normal rvs; [5min] Derivation of that the t statistic is indeed distributed as Student's T rv; [10min] derivation of the standard MultivariateNormal rv; [20min] derivation of the general MultivariateNormal rv; [10min] multivariate chfs; [10min] chf of the MultivariateNormal rv; [10min] proof that the shifted and scaled MultivariateNormal rv is another MultivariateNormal rv; [10min] Mahalanobis distance

\item[D21 - Lec 17] [10min] derivation of multivariate chf property that finds conditional distributions; [55min] derivation of the Markov tail bound and its corrollaries including Chebyshev's tail bound and Chernoff's tail bound; [10min] tail bounds for the Exponential rv; [20min] derivation of the Cauchy-Schwartz inequality and proof that correlation is bounded between -1 and +1; [20min] Jensen's inequality

\item[D22 - Lec 18] [25min] convergence in distribution, proofs for PMF and CDF convergences; [10min] Convergence in probability to a constant; [15min] weak weak law of large numbers (WLLN); [40min] convergence in law, proof that convergence in law implies convergence in probability; 

\item[D23 - Lec 19] [30min] Other useful inequalities; [20min] Continuous Mapping Theorem in one dimension; [20min] Continuous Mapping Theorem in multiple dimensions; [20min] Slutsky's theorem in one dimension

\item[D24 - Lec 20] [20min] Slutsky's theorem corrolaries e.g. sums, products; [20min] Other useful theorems for convergence in probability and distribution; [15min] law of iterated expectation; [15min] law of total variance for two rv's; [15min] general law of total variance for more than two rv's

\item[D25 - Lec 21] [45 min] theory of learning and estimation with concentration inequalities e.g. for the binomial parameter; [30min] introduction to linear programming; [35min] simplex method

\item[D26 - Lec 22] [30min] introduction to convex optimization; [40min] gradient descent; [30min] stochastic gradient descent

\item[D27 - Lec 23] [30min] Levenberg-Marquardt Algorithm (LMA); [30min] simulated annealing; [30min] evolutionary methods and particle swarm methods; [20min] Limited memory Broyden–Fletcher–Goldfarb–Shanno algorithm (L-BFGS) method

\item[D28] \inblue{Final Review}

\end{enumerate}

\textbf{This is more of a typical mathematics theory course than the rest of the data science series.} But we will still attempt to keep our eye on developing ideas and concepts for helping to make decisions in the real world. Thus we may make limited use of computation using the \texttt{R} statistical language.

\subsection*{Prerequisites}

MATH 241 (basic probability), 201 (multivariable calculus) and 231 (linear algebra) or equivalents. I expect a 241 class that covers more or less what I cover in 241. See the course homepage for links under \qu{prerequisite review}. The multivariable calculus and linear algebra we will use will be light and I will try to review those concepts in class as we need them.

\section*{Course Materials}

\paragraph{Textbook:} Introduction to Probability Theory by Hoel, Port \& Stone. This book is out of print but you can buy it used on \href{https://www.amazon.com/Introduction-Probability-Theory-Paul-Hoel/dp/039504636X/ref=sr_1_1?ie=UTF8&qid=1503515517&sr=8-1&keywords=introduction+to+probability+theory+hoel}{Amazon} for $\approx$\$20, a reasonable price (as far as textbooks go). There is no excuse not to have this book. It is \textit{required}. However, I will not ususally be teaching \qu{from the book} --- most of the material in the class comes from the lecture notes. The textbook is a way to get ``another take'' on the material. The textbook covers about only half of the material done in class (yes, sometimes we will be following the textbook page by page). For the other half, you will have to make use of other resources.

\paragraph{Computer Software:} We will also be using \texttt{R} which is a free, open source statistical programming language and console. You can download it from: \url{http://cran.mirrors.hoobly.com/}. I do not expect you to do \textit{any} programming. I will be giving you \texttt{R} code to run and expect you to interpret the results based on concepts explained during the course.

\paragraph{Calculator:} You can use a TI-84, 85, 89 or any calculator which you wish. I strongly suggest you use \href{http://www.wolframalpha.com/}{Wolfram Alpha} and its smartphone app.

\input{../../syllabi/_the650section}

\input{../../syllabi/_use_of_slack}

\input{../../syllabi/_announcements_on_slack}

\input{../../syllabi/_standard_class_meetings}

%\input{../../syllabi/_jewish_holiday_reschedule}

%\input{../../syllabi/_zoom_policies}

%\input{../../syllabi/_lecture_upload}

\section*{Homework}

\input{../../syllabi/_theory_hws_text}

\input{../../syllabi/_theory_hws_submission_text}
\input{../../syllabi/_philosophy_hws}

\input{../../syllabi/_time_spent_hws}

\input{../../syllabi/_late_hw_policy}

\input{../../syllabi/_latex_hw_bonus_policy}

\input{../../syllabi/_hw_ec_policy}

\section*{Examinations}

\input{../../syllabi/_examination_text}

\input{../../syllabi/_standard_exam_schedule}

\subsection*{Exam Policies and Materials}

\input{../../syllabi/_examination_policies}

%\input{../../syllabi/_zoom_examination_policies}

\input{../../syllabi/_standard_cheat_sheet_policy}


\input{../../syllabi/_cheating_on_exams_and_missing_exams}
\input{../../syllabi/_special_services}

\input{../../syllabi/_class_participation}

%\input{../../syllabi/_zoom_attendance}

\input{../../syllabi/_standard_grading_and_grading_policy}

\input{../../syllabi/_advanced_course_grade_distribution}

\input{../../syllabi/_grade_checking_on_gradesly}

\input{../../syllabi/_auditing_policy}

\end{document}